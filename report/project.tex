\documentclass[11pt,a4paper,twocolumn,titlepage]{article}
\usepackage[margin=1in]{geometry}
\usepackage{amsmath}
\usepackage{amsthm}
\usepackage{amssymb}
\usepackage[pdftex]{hyperref}

\setlength{\parskip}{1em}


\title{CS 6230 Project Report\\\textbf{Parallel Computation of  Centrality}}
\author{Rui Dai, Sam Olds}
\date{\today}


\begin{document}
\maketitle
\newpage


% Outline:
% 1.) Introduction
%     * What is centrality
%     * Different types
%       * Degree
%       * Closeness
%       * Betweeness
%       * Page Rank!
%       * Many More: Eigenvector centrality, Katz centrality, Percolation
%         centrality, Cross-clique centrality, Freeman Centralization
%     * Why is it important
%     * What problems does it solve
%     * How is it used
%     * The type of centrality we focused on was betweenness centrality - it's
%       a lot of shortest path calculations.
% 2.) Related work
%     * How have other people already approached this problem
%     * The best known algorithms for differenet types of centrality
%     * Algorithms we looked at
%     * Floyd-Warshall $O(V^3)$
%     * Brandes $O(VE)$
% 3.) Implementation
%     * Challenges of implementing this in parallel
%     * Challenges of finding a validly large social graph
%     * Started with an adjacency matrix and matrix multiplication
%     * Moved onto brandes algorithm using an adjacency list
%       \cite{brandes2001faster}
%     * Finally, implemented a parallelized version of brandes found here
%       \cite{bader2006parallel}
% 4.) Experimentation
%     * Began with synthetic graphs. generated graphs this way
%     * Moved to facebook graph with 4000 vertices \cite{leskovec2012learning}
%     * Ended with a gplus graph with 107000 vertices \cite{leskovec2012learning}
% 5.) Results
%     * Challenges of rendering large graphs
%     * Challenges of validating centrality
%     * Graphs of strong and weak scalability.
%     * Graphs of centrality found
% 6.) Conclusion
%     * MPI and OpenMP really sped up the processing time
% 7.) Future work
%     * We could continue working to improve the running time and finding more
%       graphs to run it on
%     * We used a shared memory model, it would be interesting to try and break
%       up the graph into a distributed memory model.



%% ============================= Introduction ============================= %%
\section{Introduction} % 1. introduction and motivation
\label{sec:intro}
%     * What is centrality
%     * Different types
%       * Degree
%       * Closeness
%       * Betweeness
%       * Page Rank!
%       * Many More: Eigenvector centrality, Katz centrality, Percolation
%         centrality, Cross-clique centrality, Freeman Centralization
%     * Why is it important
%     * What problems does it solve
%     * How is it used
%     * The type of centrality we focused on was betweenness centrality - it's
%       a lot of shortest path calculations.

In graph theory, centrality indicates the importance of a vertex in the
network. This concept is naturally applied on social network analysis. Imagine
you are producing a new product and want to find beta users. It's simple to let
users with high centrality to use and spread the news to their reachable
networks.

There are many different definitions of centrality, eg. degree centrality,
closeness centrality and betweenness centrality. In our project, we will use
the vertex betweenness centrality, which is defined as:
\[ g(v) = \sum_{s\neq v \neq t}{\frac{\delta_{st}(v)}{\delta_{st}}} \]

Betweenness centrality quantifies the number of times a node acts as a bridge
along the shortest path between two other nodes. It's not hard to imagine that
the computation of centrality is very expensive with all the operations with
shortest paths. Our goal of this project is parallel such computation and
leverage the technology of MPI/OPENMP we learned in class.

We will explain two algorithms we use for parallel centrality computation, with
experiments, result, and analysis. 


%% ============================= Related Work ============================= %%
\section{Related Work} % 2. related work
\label{sec:related-work}
%     * How have other people already approached this problem
%     * The best known algorithms for differenet types of centrality
%     * Algorithms we looked at
%     * Floyd-Warshall $O(V^3)$
%     * Brandes $O(VE)$

Centrality was first introduced as a measure for quantifying the control of a
human on the communication between other humans in a social network by Linton
Freeman\cite{burt2009structural} In his conception. Ever since, the concept has
drawn many interests in cross-indiscipline area like network analysis and
social science.

Betweenness centralities in a graph involve calculating the shortest paths
between all pairs of vertices on a graph, which requires $\Theta(V^3)$ time
with the Floyd–Warshall\cite{Cormen:2001:IA:580470} algorithm. On sparse
graphs, Johnson's\cite{johnson1977efficient} algorithm may be more efficient,
taking $O(V^2 \log V + V E)$ time. In the case of unweighted graphs the
calculations can be done with Brandes' algorithm\cite{brandes2001faster} which
takes $\Theta(VE)$ time. Normally, these algorithms assume that graphs are
undirected and connected with the allowance of loops and multiple edges. When
specifically dealing with network graphs, often graphs are without loops or
multiple edges to maintain simple relationships (where edges represent
connections between two people or vertices). In this case, using Brandes'
algorithm will divide final centrality scores by 2 to account for each shortest
path being counted twice.

Breadth first search is also a great part in centrality computation as for the
shortest path part. Parallel BFS algorithms have 


%% ============================= Data ============================= %%
\section{Implementation} % 3. methods/algorithms
\label{sec:data}
%     * Challenges of implementing this in parallel
%     * Challenges of finding a validly large social graph
%     * Started with an adjacency matrix and matrix multiplication
%     * Moved onto brandes algorithm using an adjacency list
%       \cite{brandes2001faster}
%     * Finally, implemented a parallelized version of brandes found here
%       \cite{bader2006parallel}

\subsection{Representation}
In our work, we represent graph in sparse adjacency matrix. 


\subsection{Graph Generation}


%% ============================= Methods ============================= %%
\section{Experimentation} % 3. methods/algorithms
\label{sec:methods}
%     * Began with synthetic graphs. generated graphs this way
%     * Moved to facebook graph with 4000 vertices \cite{leskovec2012learning}
%     * Ended with a gplus graph with 107000 vertices \cite{leskovec2012learning}

\subsection{Simple Solution}

\subsection{Method Based on Brandes'}
Sam


%% ============================= Results ============================= %%
\section{Results} % 4. experiments and results
\label{sec:results}
%     * Challenges of rendering large graphs
%     * Challenges of validating centrality
%     * Graphs of strong and weak scalability.
%     * Graphs of centrality found

Sam

%% ============================= Conclusion ============================= %%
\section{Conclusion} % 5. conclusions and future directions
\label{sec:conclusion}
%     * MPI and OpenMP really sped up the processing time



%% ============================= Future Work ============================= %%
\section{Future Work}
\label{sec:future-work}
%     * We could continue working to improve the running time and finding more
%       graphs to run it on
%     * We used a shared memory model, it would be interesting to try and break
%       up the graph into a distributed memory model.



%% ============================= Bibliography ============================= %%
\newpage
\bibliographystyle{acm}
\bibliography{references}



\end{document}
